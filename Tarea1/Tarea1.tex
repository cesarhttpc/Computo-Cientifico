\documentclass[a4paper, 11pt]{article}
\usepackage{comment}
\usepackage{lipsum} 
\usepackage{fullpage} %cambiar margen
\usepackage[a4paper, total={7in, 10in}]{geometry}

\usepackage{amssymb,amsthm} 
\usepackage{amsmath}
\newtheorem{theorem}{Theorem}
\newtheorem{corollary}{Corollary}
\usepackage{graphicx}
\usepackage{tikz}
\usetikzlibrary{arrows}
\usepackage{verbatim}
%\usepackage[numbered]{mcode}
\usepackage{float}
\usepackage{tikz}
\usetikzlibrary{shapes,arrows}
\usetikzlibrary{arrows,calc,positioning}
\usepackage{mathpazo} %tipo de letra 
\usepackage[utf8]{inputenc} %codificación
\usepackage[T1]{fontenc} %digitación de tildes y ñ
\usepackage[spanish]{babel} %paquete de soporte español

\tikzset{
	block/.style = {draw, rectangle,
		minimum height=1cm,
		minimum width=1.5cm},
	input/.style = {coordinate,node distance=1cm},
	output/.style = {coordinate,node distance=4cm},
	arrow/.style={draw, -latex,node distance=2cm},
	pinstyle/.style = {pin edge={latex-, black,node distance=2cm}},
	sum/.style = {draw, circle, node distance=1cm},
}
\usepackage{xcolor}
\usepackage{mdframed}
\usepackage[shortlabels]{enumitem}
\usepackage{indentfirst}
\usepackage{hyperref}

\usepackage{listings}
\lstset{literate=
  {á}{{\'a}}1
  {é}{{\'e}}1
  {í}{{\'i}}1
  {ó}{{\'o}}1
  {ú}{{\'u}}1
  {Á}{{\'A}}1
  {É}{{\'E}}1
  {Í}{{\'I}}1
  {Ó}{{\'O}}1
  {Ú}{{\'U}}1
  {ñ}{{\~n}}1
  {ü}{{\"u}}1
  {Ü}{{\"U}}1
}

\lstdefinestyle{customc}{
  belowcaptionskip=1\baselineskip,
  breaklines=true,
  frame=L,
  xleftmargin=\parindent,
  language=Python,
  showstringspaces=false,
  basicstyle=\footnotesize\ttfamily,
  keywordstyle=\bfseries\color{green!40!black},
  commentstyle=\itshape\color{purple!40!black},
  identifierstyle=\color{blue},
  stringstyle=\color{orange},
}

\lstdefinestyle{customasm}{
  belowcaptionskip=1\baselineskip,
  frame=L,
  xleftmargin=\parindent,
  language=[x86masm]Assembler,
  basicstyle=\footnotesize\ttfamily,
  commentstyle=\itshape\color{purple!40!black},
}

\lstset{escapechar=@,style=customc}



\renewcommand{\thesubsection}{\thesection.\alph{subsection}}

\newenvironment{problem}[2][Ejercicio]
{ \begin{mdframed}[backgroundcolor= red!50] \textbf{#1 #2} \\}
	{  \end{mdframed}}

% Define solution environment
\newenvironment{solution}
{\textcolor{blue}{\textbf{\textit{Solución:\\\noindent}}}}


\renewcommand{\qed}{\quad\qedsymbol}

% \\	
\begin{document}
	\noindent
	%%%%%%%%%%%%%%%%%%%%%%%%%%%%%%%%%%%%
	
	\begin{minipage}[b][1.2cm][t]{0.8\textwidth}
		\large\textbf{César Isaí García Cornejo} \hfill \textbf{Tarea 1}  \\
		cesar.cornejo@cimat.mx \hfill \\
		\normalsize Computo Científico \hfill Semestre 3\\
	\end{minipage}
	
	\hspace{14.4cm}
	\begin{minipage}[b][0.03cm][t]{0.12\linewidth}
		
		\vspace{-2.2cm}
		%%%La Ruta depeendera de donde este alojado el main y la imagen
		\includegraphics[scale=0.3]{Figures/EscudoCimat.png}
	\end{minipage}
	
	\noindent\rule{7in}{2.8pt}
	
	%%%%%%%%%%%%%%%%%%%%%
	%%%%%%%%%%%%%%%%%%%%%%%%%%%%%%%%%%%%%%%%%%%%%%%%%%%%%%%%%%%%%%%%%%%%%%%%%%%%%%%%%%%%%%%%%%%%%%%%%%%%%%%%%%%%%%%%%%%
	% Problem 1
	%%%%%%%%%%%%%%%%%%%%%%%%%%%%%%%%%%%%%%%%%%%%%%%%%%%%%%%%%%%%%%%%%%%%%%%%%%%%%%%%%%%%%%%%%%%%%%%%%%%%%%%%%%%%%%%%%%%%%%%%%%%%%%%%%%%%%%%%
	\setlength{\parskip}{\medskipamount}
	\setlength{\parindent}{0pt}
 
\begin{problem}{1}
Implementar los algoritmos de \textit{Backward} y \textit{Forward substitution}.
\end{problem}

\begin{solution}

\end{solution}
	
\begin{problem}{2}
    Implementar el algoritmo de eliminación Gaussiana con pivoteo parcial LUP, 21.1 del Trefethen (p. 160).
\end{problem}
    
\begin{problem}{3}
    Dar la descomposición LUP para una matriz aleatoria de entradas U(0,1) de tamaño 5$\times 5$, y para la matriz
    \begin{align}
        A = \begin{pmatrix}
            1 &0  &0  &0 &1 \\ 
            -1 &1  &0  &0 &1 \\ 
            -1 &1  &1 &0  &1 \\ 
            -1 &-1  &-1  &1 &1\\
            -1 & -1 & -1 & -1 &1 
            \end{pmatrix}
    \end{align}
\end{problem}



\begin{problem}{4}
    Usando la descomposición LUP anterior, resolver el sistema de la forma
    \begin{align}
        Dx = b 
    \end{align}
    donde $D$ son las matrices del problema 3, para 5 diferentes $b$ aleatorios con entradas $U(0,1)$. Verificando si es posible o no resolver el sistema.
\end{problem}


\begin{problem}{4}
    Implementar el algoritmo de descomposición de Cholesky 23.1 del Trefethen (p. 175).
\end{problem}

\begin{problem}{5}
    Comparar la complejidad de su implementación de los algoritmos de factorización de Cholesky y LUP mediante la medición de los tiempos que tardan con respecto a la descomposición de una matriz aleatoria hermitiana definida positiva. Graficar la comparación.
\end{problem}
\end{document}