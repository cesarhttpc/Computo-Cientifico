\documentclass[a4paper, 11pt]{article}
\usepackage{comment}
\usepackage{lipsum} 
\usepackage{fullpage} %cambiar margen
\usepackage[a4paper, total={7in, 10in}]{geometry}

\usepackage{amssymb,amsthm} 
\usepackage{amsmath}
\newtheorem{theorem}{Theorem}
\newtheorem{corollary}{Corollary}
\usepackage{graphicx}
\usepackage{tikz}
\usetikzlibrary{arrows}
\usepackage{verbatim}
%\usepackage[numbered]{mcode}
\usepackage{float}
\usepackage{tikz}
\usetikzlibrary{shapes,arrows}
\usetikzlibrary{arrows,calc,positioning}
\usepackage{mathpazo} %tipo de letra 
\usepackage[utf8]{inputenc} %codificación
\usepackage[T1]{fontenc} %digitación de tildes y ñ
\usepackage[spanish]{babel} %paquete de soporte español

\tikzset{
	block/.style = {draw, rectangle,
		minimum height=1cm,
		minimum width=1.5cm},
	input/.style = {coordinate,node distance=1cm},
	output/.style = {coordinate,node distance=4cm},
	arrow/.style={draw, -latex,node distance=2cm},
	pinstyle/.style = {pin edge={latex-, black,node distance=2cm}},
	sum/.style = {draw, circle, node distance=1cm},
}
\usepackage{xcolor}
\usepackage{mdframed}
\usepackage[shortlabels]{enumitem}
\usepackage{indentfirst}
\usepackage{hyperref}

\usepackage{listings}
\lstset{literate=
  {á}{{\'a}}1
  {é}{{\'e}}1
  {í}{{\'i}}1
  {ó}{{\'o}}1
  {ú}{{\'u}}1
  {Á}{{\'A}}1
  {É}{{\'E}}1
  {Í}{{\'I}}1
  {Ó}{{\'O}}1
  {Ú}{{\'U}}1
  {ñ}{{\~n}}1
  {ü}{{\"u}}1
  {Ü}{{\"U}}1
}

\lstdefinestyle{customc}{
  belowcaptionskip=1\baselineskip,
  breaklines=true,
  frame=L,
  xleftmargin=\parindent,
  language=Python,
  showstringspaces=false,
  basicstyle=\footnotesize\ttfamily,
  keywordstyle=\bfseries\color{green!40!black},
  commentstyle=\itshape\color{purple!40!black},
  identifierstyle=\color{blue},
  stringstyle=\color{orange},
}

\lstdefinestyle{customasm}{
  belowcaptionskip=1\baselineskip,
  frame=L,
  xleftmargin=\parindent,
  language=[x86masm]Assembler,
  basicstyle=\footnotesize\ttfamily,
  commentstyle=\itshape\color{purple!40!black},
}

\lstset{escapechar=@,style=customc}



\renewcommand{\thesubsection}{\thesection.\alph{subsection}}

\newenvironment{problem}[2][Ejercicio]
{ \begin{mdframed}[backgroundcolor= red!50] \textbf{#1 #2} \\}
	{  \end{mdframed}}

% Define solution environment
\newenvironment{solution}
{\textcolor{blue}{\textbf{\textit{Solución:\\\noindent}}}}


\renewcommand{\qed}{\quad\qedsymbol}

% \\	
\begin{document}
	\noindent
	%%%%%%%%%%%%%%%%%%%%%%%%%%%%%%%%%%%%
	
	\begin{minipage}[b][1.2cm][t]{0.8\textwidth}
		\large\textbf{César Isaí García Cornejo} \hfill \textbf{Tarea 6}  \\
		cesar.cornejo@cimat.mx \hfill \\
		\normalsize Computo Científico \hfill Semestre 3\\
	\end{minipage}
	
	\hspace{14.4cm}
	\begin{minipage}[b][0.03cm][t]{0.12\linewidth}
		
		\vspace{-2.2cm}
		%%%La Ruta dependerá de donde este alojado el main y la imagen
		\includegraphics[scale=0.3]{Figures/EscudoCimat.png}
	\end{minipage}
	
	\noindent\rule{7in}{2.8pt}
	
	%%%%%%%%%%%%%%%%%%%%%
	%%%%%%%%%%%%%%%%%%%%%%%%%%%%%%%%%%%%%%%%%%%%%%%%%%%%%%%%%%%%%%%%%%%%%%%%%%%%%%%%%%%%%%%%%%%%%%%%%%%%%%%%%%%%%%%%%%%
	% Problem 1
	%%%%%%%%%%%%%%%%%%%%%%%%%%%%%%%%%%%%%%%%%%%%%%%%%%%%%%%%%%%%%%%%%%%%%%%%%%%%%%%%%%%%%%%%%%%%%%%%%%%%%%%%%%%%%%%%%%%%%%%%%%%%%%%%%%%%%%%%
	\setlength{\parskip}{\medskipamount}
	\setlength{\parindent}{0pt}
%/////////// Ejercicio 1 /////////////////

    
\begin{problem}{1} 
    Simular $n=5$ y $n=40$ v.a. Bernoulli $Be(1/3)$; sea $r$ el número de éxitos en cada caso.
\end{problem}

\begin{solution} 
    
\end{solution}

\begin{problem}{2} 
    Implementar el algoritmo Metropolis-Hastings para simular de la posterior
    \begin{align*}
        f(p|\bar{x }) \propto p^r (1-p )^{n-r } cos(\pi p ) I_{[0,\frac{1}{2}]}(p),
    \end{align*}
    con los dos casos de $n$ y $r$ de arriba. Para ello poner la propuesta $(p'|p ) = p' \sim Bete(r+1,n-r+1)$ y la distribución inicial de la cadena $\mu \sim U(0,\frac{1}{2})$
\end{problem}

\begin{solution} 
  
\end{solution}

\begin{problem}{3} 
    Argumentar porque la cadena es $f$-irreducible y porque es ergódica. Implementar el algoritmo con los datos descritos y discutir los resultados.
\end{problem}

\begin{solution} 
  
\end{solution}

\begin{problem}{4} 
    Implementar el algoritmo Metropolis-Hastings con la posterior de arriba tomando una propuesta diferente.
\end{problem}

\begin{solution} 
  
\end{solution}

% \begin{thebibliography}{9}

%     \bibitem{Casella}
%     Robert, C. P., Casella, G., and Casella, G. (1999). Monte Carlo statistical methods (Vol. 2). New York: Springer.
    
%     % \bibitem{Scipy.stats}
%     % https://github.com/scipy/scipy/blob/v1.11.3/scipy/stats/_continuous_distns.py#L9766-L9946
    
    
% \end{thebibliography}
      




    \end{document}